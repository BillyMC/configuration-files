\chapter{绪论}
\section{选题背景}
\subsection{IEEE 802.15.4 协议的提出}

\subsection{IEEE 802.15.4 协议的研究现状}

\section{选题意义}
\section{论文主要工作及结构}

本论文的结构为:
\cvitem{0.132}{0.83}{第一章} {绪论; 阐述选题背景与选题意义,以及论文的主要工作和结构。}
\cvitem{0.132}{0.83}{第二章} {IEEE 802.15.4 标准概述;介绍 IEEE 802.15.4 标准,对 IEEE 802.15.4 物理层以及 MAC 层协议进行详细介绍。}
\cvitem{0.132}{0.83}{第三章} {系统分析;提出系统的需求分析,介绍系统所要实现的功能。}
\chapter{IEEE 802.15.4 标准}
\section{IEEE 802.15.4 网络}
IEEE 802.15.4 网络是指在一个 POS (personal operating space) 范围内使用相同的无线信道并通过 IEEE 802.15.4 标准相互通信的一组设备的集合,即 LR-WPAN 网络。设备(device)是 IEEE 802.15.4 网络最基本的组成部分。IEEE 802.15.4 网络中支持两种设备,一种是全功能设备(full-devic, FFD),另一种是精简功能设备(reduced-device, RFD)。全功能设备可以在三种模式下工作,即 PAN Coordinator、Coordinator 和普通的网络设备,网络运行期间可以动态地在这三种模式之间进行切换。全功能设备可以和精简功能设备或者其他全功能设备通信,同时可以控制网络的拓扑结构。精简功能设备不能和全功能设备进行通信,只能和全功能设备通信,或者通过全功能设备向外发送数据,并且精简功能设备不能控制网络的拓扑结构。精简功能设备通常用于简单的控制应用,传输的数据量较少,只需占用很少的传输资源和通信资源。因此,精简设备的实现方案一般比较廉价。

IEEE 802.15.4 标准规定的网络具有如下特点:
\begin{compactitem}
\item 在不同的载波频率下实现 20kbps、40kbps 和 250kbps 三种不同的传输速率;
\item 支持星型和点对点两种网络拓扑结构;
\item 具有 16 位和 64 位两种地址格式,其中 64 位地址是全球唯一的扩展地址;
\item 支持冲突避免的载波多路侦听技术 (carrier sense multiple access with collision avoidance, CSMA/CA);
\item 支持确认(ACK)机制,保证传输可靠性。
\end{compactitem}

\section{物理层}
\section{MAC层}
\section{本章小节}

\chapter{系统分析}
\section{需求分析}
\subsection{功能需求}
\subsection{工作流程}

\section{系统实现目标}

\section{本章小节}


\chapter{系统设计}
\section{PC端软件设计}
\section{分析仪设备端软件设计}
\section{软件通信}
\section{本章小节}
\chapter{系统实现}

\section{本章小节}
\subsection{测试章节}

\chapter{结束语}


%%% Local Variables: 
%%% mode: latex
%%% TeX-master: t
%%% End: 
