\chapter{绪论}
\section{选题背景}
\subsection{IEEE 802.15.4 协议的提出}

随着通信技术的飞速发展,通信网络从布线网络发展到了无线网络。基于 IEEE 802.11 的无线局域网标准已成为无线局网的主要标准,被广泛应用于家庭、办公以及公共场合。然而,IEEE 802.11 协议的组网成本较高,主要用于组建计算机局域网络并支持大量数据的传输。而随着电子信息技术的发展,手持设备以及其他移动设备的普及,人们提出了短距离通信的需求,从而出现了个人区域网络(Personal Area Network, PAN) 和无线个人区域网络(Wireless Personal Area Network, WPAN)。其中的WPAN以简单方便、灵活安全、低成本、低功耗为主要特点。

1998年3月,IEEE 802.15工作组成立,该工作组致力于WPAN网络的物理层(PHY)和媒体访问层(MAC)的标准工作。为了满足低功耗、低成本的无线网络需求,IEEE 标准委员会于2000年12月正式批准并成立IEEE 802.15.4工作组,负责针对低速无线个人区域网络(low-rate wireless personal area network, LR-WPAN)制定标准,旨在为个人区域网内不同设备的低速互连提供统一标准。

2003年IEEE 802.15.4协议标准被批准并正式提出。

\subsection{IEEE 802.15.4 协议的研究现状}
IEEE 802.15.4 是 Zigbee, WirelessHART 和 Miwi 规范的基础。其中,来自全球130多家会员组成的 Zigbee 联盟致力于推动 Zigbee 的应用及发展,使得目前 Zigbee 在数字家庭、工业、智能交通等领域有着十分广泛的应用,并面临着大好发展前景。

当前,将IP协议引入无线通信网络的设想与实现已成为无线个域网的热点问题。IP对内存以及带宽要求较高,如何降低其运行环境以适应微控制器及低功率无线连接是实现将IP协议引入无线通信网络的难点问题。目前,基于IEEE 802.15.4 协议实现 IPv6 通信的 IETF 6LoWPAN 草案标准正致力于解决上述问题。

\section{选题意义}
利用协议分析可以对网络进行故障排除,对网络进行测试以及优化等,为网络的建设与维护提供了基础。

本课题是基于 IEEE 802.15.4 进行协议分析的设计与实现,通过为该协议的物理层与 MAC 层进行专门数据分析,对所处环境中的802.15.4网络进行测试评估、流量监控、优化与维护。为研究与实现 6LoWPAN 网络等工作提供基础工具。
\section{论文主要工作及结构}
本论文在对 IEEE 802.15.4 协议进行研究分析,深入理解 IEEE 802.15.4 物理层以及 MAC 层结构与特点的基础上,对 IEEE 802.15.4 协议进行协议分析仪的软件设计与实现,阐述从系统分析、系统设计到系统实现的整个过程。

\pagebreak[4]

本论文的结构为:
\cvitem{0.132}{0.83}{第一章} {绪论; 阐述选题背景与选题意义,以及论文的主要工作和结构。}
\cvitem{0.132}{0.83}{第二章} {IEEE 802.15.4 标准概述;介绍 IEEE 802.15.4 标准,对 IEEE 802.15.4 物理层以及 MAC 层协议进行详细介绍。}
\cvitem{0.132}{0.83}{第三章} {系统分析;提出系统的需求分析,介绍系统所要实现的功能。}
\chapter{IEEE 802.15.4 标准}
\section{IEEE 802.15.4 网络}
IEEE 802.15.4 网络是指在一个 POS (personal operating space) 范围内使用相同的无线信道并通过 IEEE 802.15.4 标准相互通信的一组设备的集合,即 LR-WPAN 网络。设备(device)是 IEEE 802.15.4 网络最基本的组成部分。IEEE 802.15.4 网络中支持两种设备,一种是全功能设备(full-devic, FFD),另一种是精简功能设备(reduced-device, RFD)。全功能设备可以在三种模式下工作,即 PAN Coordinator、Coordinator 和普通的网络设备,网络运行期间可以动态地在这三种模式之间进行切换。全功能设备可以和精简功能设备或者其他全功能设备通信,同时可以控制网络的拓扑结构。精简功能设备不能和全功能设备进行通信,只能和全功能设备通信,或者通过全功能设备向外发送数据,并且精简功能设备不能控制网络的拓扑结构。精简功能设备通常用于简单的控制应用,传输的数据量较少,只需占用很少的传输资源和通信资源。因此,精简设备的实现方案一般比较廉价。

IEEE 802.15.4 标准规定的网络具有如下特点:
\begin{compactitem}
\item 在不同的载波频率下实现 20kbps、40kbps 和 250kbps 三种不同的传输速率;
\item 支持星型和点对点两种网络拓扑结构;
\item 具有 16 位和 64 位两种地址格式,其中 64 位地址是全球唯一的扩展地址;
\item 支持冲突避免的载波多路侦听技术 (carrier sense multiple access with collision avoidance, CSMA/CA);
\item 支持确认(ACK)机制,保证传输可靠性。
\end{compactitem}

\section{物理层}
\section{MAC层}
\section{本章小节}

\chapter{系统分析}
\section{需求分析}
\subsection{功能需求}
\subsection{工作流程}

\section{系统实现目标}

\section{本章小节}


\chapter{系统设计}
\section{PC端软件设计}
\section{分析仪设备端软件设计}
\section{软件通信}
\section{本章小节}
\chapter{系统实现}

\section{本章小节}
\subsection{测试章节}

\chapter{结束语}


%%% Local Variables: 
%%% mode: latex
%%% TeX-master: t
%%% End: 
